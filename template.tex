% -*- coding: utf-8-unix; mode: latex; fill-column: 70 -*-

\documentclass{stabs2021}

% Choose your dialect of English for correct hyphenation patterns
% List of languages in documentation on ctan.org/pkg/babel
% Section 1.27 "Languages supported by babel with ldf files"
\usepackage[british]{babel}

% Recommended for mathematical typesetting
\usepackage{amsmath,amssymb}

\begin{document}

\title{Guidelines for Preparation of Manuscripts}

\author{
  Firstname Surname, \textit{Affiliation},
  \href{mailto:test@example.com}{test@example.com}
  \and
  Firstname Surname, \textit{Other affiliation},
  \href{mailto:test@example.com}{test@example.com}
}

% Don't write out any text before this environment
% It starts a new page and typesets its content in one-column mode
\begin{frontmatter}
  \maketitle

  \begin{abstract}
    Authors of papers have to type these in a form suitable for direct
    photographic reproduction by the publisher. In order to ensure
    uniform style throughout the volume, all the papers have to be
    prepared strictly according to the instructions set below, which
    essentially follow the ITTC format.

    The abstract should be a brief description of the scope of the
    paper, not exceeding 100 words in length.
  \end{abstract}

  \keywords{at least 3 suitable keywords for indexing purposes}
\end{frontmatter}

\section{Introduction}

Please note that this section and all subsequent sections and
subsections are numbered.  All main headings and sub-headings should
be typed in bold as shown below.

\section{Introductory Information for Authors}

\subsection{Word Processor}

It is highly recommended to generate the paper on a personal computer
or workstation using the Microsoft Word program (version XP or
later). Other computers and word processing programs may be used,
although in this case, additional work by the proceedings editor is
required on the source file, in order to convert it to the present
format.

\subsection{Digital Copy}

Authors should upload their papers through the STAB\&S 2021 Conference
web site (\url{http://www.stability-and-safety-2021.org/}) Please
select ``Submit Contribution'' and then, at the field ``Conference
Track/Type of Submission'', select ``Full Paper''.

The maximum paper length, after formatting to the STAB\&S 2021
specification, is 10 pages. Whilst slightly longer papers can still be
accommodated, authors are advised to do their best in order to respect
this page limit.

\subsection{Liability}

Authors are responsible for obtaining security approval for
publication from employers or authorities, where necessary. If they so
wish, authors should include a disclaimer at the end of the paper
stating that the opinions expressed are those of the author and not
those of the company or organization that they are representing.

\section{Page Format}

It is required that the papers be prepared on A4 page format (210 mm
\(\times\) 297 mm). All dimensions concerning the page layout are
given in millimetres. The Microsoft Word program allows the user to
specify the configuration using these metric dimensions.

\subsection{Top and Bottom Margins}

\paragraph{General.}

The bottom margin for all pages is 20 mm. This is the distance between
the bottom of the last line of text and the bottom edge of the sheet
of paper. The top margin for all pages is 32 mm. The top margin is
measured from the top edge of the paper to the top of the first line
of text.

\subsection{Columns and Side Margins}

The manuscript is required to be prepared in two-column format. The
column width is to be 81 mm and the spacing between the columns is to
be 8 mm. The left side margin is to be 20 mm. The corresponding right
hand margin is 20 mm.

\section{Manuscript Format Conventions}

\paragraph{Font.}

The text font must be 12 point ``Times New Roman'' type (the most
common type fonts available on laser printers). Greek letters
appearing in equations or text (such as \(\alpha\), \(\beta\),
\(\gamma\)) should also be 12 point and set in the standard ``Symbol''
type font. The vertical line spacing should be 14 points (equating a
line height of 4.95 mm).

\paragraph{Justification.}

The text (but not the headings) should also be justified so that it
fills up the space in the columns exactly. Hyphenation (a standard
feature on most word processors) should be used to break the text so
that it nearly fills each line. The appearance of the volume will be
compromised if the text has not been hyphenated, since justification
can lead to some lines with large spaces between the words.

\paragraph{Paragraphs.}

All paragraphs are to be indented 6 mm. Paragraphs should not be
separated from any blank line. The proper distance between paragraphs
(14 pt) is to be assigned through the properties of the paragraph
itself.

All features described in the three paragraphs above are
automatically implemented in typing text using the ``Paragraph-1''
style included in the present format file.

\subsection{Headings}

Headings and subheadings should appear throughout the paper to divide
the subject matter into logical parts and to emphasize the major
elements and considerations. Each section may have subheadings, as
detailed below. Parts or sections should be numbered with one digit
(X.) for the main headings, two digits (X.X) for the first
subheadings. Further subheadings should not be numbered.

Headings should not appear at the bottom of a column, if there is no
text following them in the same column. If the normal flow of text
causes this to occur, the editor may try to prevent it by formatting
one or more neighbouring paragraphs with ``Paragraph-1+'' so that the
heading appears at the top of the next column.

\paragraph{Major Headings.}

Major headings should appear in bold capital letters and aligned flush
with the left-hand margin of the column. Space corresponding to two
blank lines should be left above the major heading and to one blank
line below it. All these features are automatically implemented if the
``Major-Heading'' style is used, which is included in the present
format file. The only exception to the standard format for major
headings is represented by the first major heading located at the top
of the left column in the first page (``INTRODUCTION'' in the present
sample file), for which the ``Major-Heading-Top'' style is used, to
avoid leaving at the column top the space corresponding to the two
blank lines.

\paragraph{Subheadings.}

Subheadings should appear in bold letters with the initial letter of
each word capitalized and aligned flush with the left-hand margin of
the column. Space corresponding to two blank lines should be left
above the major heading and to one blank line below it. All these
features (apart from typing capitalised initial letters for each word)
are automatically implemented if the ``Subheading'' style is used,
which is included in the present format file. The only exception to
the standard format for subheadings is represented by a subheading
that may happen to be located at the top of the right column in the
first page. In this case the ``Subheading-Top'' style can be used, to
avoid leaving at the column top the space corresponding to the two
blank lines.

\paragraph{Sub-Subheadings.}

Sub-subheadings should be indented 6 mm and appear in underlined
letters, with the initial letter of each word capitalised. The
sub-subheading should be followed by a period, two spaces and the
text. One line of space should be left above the sub-subheading. All
these features (apart from typing capitalised initial letters for
each word) are automatically implemented if the ``Subheading'' style
is used, which is included in the present format file.

\subsection{Footnotes}

Footnotes are references with superscript numerals and are to be
numbered consecutively from 1 to the end of the
paper\footnote{Footnotes should appear in Times New Roman font in
  smaller 10 point type.}. Footnotes should appear at the bottom of
the column in which they are referenced or, if necessary, at the
bottom of the next column on the same page. A solid line is used in
this format to separate the footnotes from the rest of the
text\footnote{The line above the footnote is optional, but it does
  help to keep the footnoes separate from the main body of
  text}. Whenever possible, the use of footnotes should be avoided.

[...]

\section{Mathematics}

Equations should be numbered consecutively beginning with (1) to the
end of the report, including any appendices. The number should be
enclosed in parentheses (as shown above) and set flush right in the
column on the same line as the first line of the equation. This is the
number that should be used when referring to equations within the
text.

\subsection{Printing}

Equations should be typed using the standard equation editor
available on Microsoft Word program. Vector quantities should appear
in bold lower case letters and tensor quantities in bold upper case
letters. For instance, the Bernoulli’s equation is
\begin{equation}
  \frac{\partial\phi}{\partial t} + \frac12 |\nabla \phi|^2 + \frac P\rho + gy = C(t)
\end{equation}

The continuity and Navier-Stokes equations are
\begin{equation}
  \nabla\cdot u = 0
\end{equation}
\begin{equation}
  \frac{\partial u}{\partial t} + u\cdot \nabla u = -\frac1\rho \nabla p - g + \nu \nabla^2 u
\end{equation}

In all mathematical expressions and analyses, any symbols (and the
units in which they are used) not previously defined in the
nomenclature should be explained. An extra line of space is to be
left above and below a displayed equation or formula. To achieve this,
in the present format the equations have been included in tables, so
that blank table rows can be exploited. However, this is just a
suggestion, and different techniques can be used.

\section{Graphic Material}

\subsection{General Guidelines}

All Figures (graphs, line drawings, photographs, etc.) should be
numbered consecutively and have a caption consistent of the figure
number and a brief title (the “Figure-Cap” style can be used to
automatically number the figures, as shown by the example). This will
also allow automatic referencing of the figure within text. Many
different file formats are accepted (or created) by Microsoft Word,
to be inserted and formatted along with text.

All illustrations should be clearly referenced in the text; these
should be placed in the main body of the text.  Figures should be
produced electronically in .jpg or .tif formats. Save another copy of
each individual graphic in separate files (Fig1, Fig2, etc.) in
addition to the complete manuscript file. The resolution should be at
least 300dpi, and preferably above 500dpi. Thin line computer prints
of curves, etc. must be thickened. Figures may be in colour (in
consideration for the CD version in Acrobat) or in black and
white. Please note that the hardcopy version of the proceedings will
be printed in black only. If you use colour graphics please check the
graphic in black and white to see if shading or hatching is needed.

\subsection{Placement}

Depending on size, the artwork, graphs, charts, line drawings,
sketches and diagrams, etc. should be positioned either within one
column or spanning both columns (in this case, a frame should be used
to include the figure or table and the caption, similar to that used
for the paper title). If the figure spans two columns, the caption
should be properly centred (the “Figure-Cap-C” style may be used in
this case). Captions for figures are placed below the figures while
table captions (“Table-Cap” and “Table-Cap-C” may be used for the case
of Justified and Centred Captions, respectively) are placed above the
tables. Space corresponding to a blank line should be provided above
and below figures and their captions.

\section{References}

\subsection{Text Citation}

Within the text, references should be cited by giving the last name of
the author(s) and the year of publication of the reference. The year
should always be enclosed by parentheses, while enclosing the name of
the author(s) within the same parentheses depends on the context. Some
examples using the sample references listed below are illustrated
hereafter:

\dots\ It was shown by \citet{kwon1981prediction} that numerical
integration of the Navier-Stokes equations can be successfully
performed for low Reynolds numbers. \dots

\dots\ Heat transfer in a duct is improved substantially by using
small, rectangular protuberances \citep{sparrow1980forced}. \dots

\dots\ Convection of this type is treated in several sources
\citep{lee1982structure, sparrow1980fluid, tung1982evaporative} \dots

[...]

\bibliography{bibtex}

\end{document}
